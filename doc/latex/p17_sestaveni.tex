\chapter{17 Sestavení}
\hypertarget{p17_sestaveni}{}\label{p17_sestaveni}\index{17 Sestavení@{17 Sestavení}}
Projekt byl testován na Ubuntu 20.\+04 -\/ 24.\+04, Visual Studio 2017, 2019. Projekt vyžaduje 64 bitové sestavení. Projekt využívá build systém \href{https://cmake.org/}{\texttt{ CMAKE}}. CMake je program, který na základně konfiguračních souborů "{}\+CMake\+Lists.\+txt"{} vytvoří "{}makefile"{} v daném vývojovém prostředí. Dokáže generovat makefile pro Linux, mingw, solution file pro Microsoft Visual Studio, a další.~\newline
 Postup Linux\+: 
\begin{DoxyCode}{0}
\DoxyCodeLine{\#\ stáhnout\ projekt}
\DoxyCodeLine{unzip\ izgProject.zip\ -\/d\ izgProject}
\DoxyCodeLine{cd\ izgProject/build}
\DoxyCodeLine{cmake\ ..}
\DoxyCodeLine{make\ -\/j8}
\DoxyCodeLine{./izgProject}
\DoxyCodeLine{./izgProject\ -\/h}

\end{DoxyCode}
 Posup na Windows\+:
\begin{DoxyEnumerate}
\item stáhnout projekt
\item rozbalit projekt
\item jděte do složky build/
\item ve složce build pusťte cmake-\/gui ..
\item pokud nevíte jak, tak pusťte cmake-\/gui a nastavte "{}\+Where is the source code\+:"{} na složku s projektem (obsahuje CMake\+Lists.\+txt)
\item a "{}\+Where to build the binaries\+: "{} na složku build
\item configure
\item generate
\item Otevřete vygenerovnou Microsoft Visual Studio Solution soubor. 
\end{DoxyEnumerate}